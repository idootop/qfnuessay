\section{引言}

课程论文(设计)是重要的实践教学环节, 是根据人才培养方案, 依托专业课程, 在教师指导下对学生进行的初步科研训练. 课程论文(设计)旨在巩固、深化和拓展学生的理论知识与专业技能, 培养学生综合运用理论知识分析和解决实际问题的能力, 培养学生理论联系实际的工作作风和严肃认真、实事求是的科学态度, 培养学生获取、处理信息的能力和语言文字表达能力.

目前, 学校的教务部门提供了本科生课程论文的Word模板, 规定了论文写作的一系列格式. 然而, 在数学论文的排版方面, \LaTeX 系统比Word更有优势. \LaTeX 是\TeX 排版引擎的封装, 具有方便而强大的数学公式排版能力, 很容易生成复杂的专业排版元素, 如脚注、交叉引用、参考文献、目录等. 绝大多数时候, \LaTeX 用户只需专注于一些组织文档结构的基础命令, 无需(或很少)操心文档的版面设计. 为了配合数学专业本科生的课程论文写作, 本文作者按照曲阜师范大学本科生课程论文格式的要求, 制作了对应的\LaTeX 模板.

本文结构如下: 第2节, 我们给出模板的使用说明, 包含一些重要的注意事项. 第3节, 我们列举了一些常用的\LaTeX 使用例子.

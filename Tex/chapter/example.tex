\section{\LaTeX 示例}

读者应仔细查看这部分的tex源文件内容.

\subsection{定理环境}

下面是一个定义. 将对应的\LaTeX{}环境命令里的definition换成theorem, lemma, proposition, corollary,example, remark,就得到定理、引理、命题、推论、例、注等).
\begin{definition}[\cite{Liang-Xu-Auto18}]\label{Def: positive def matrix}
  %------\begin{definition}[引用的定义所在的参考文献的标签] \label{给这个定义设置一个标签}
  %----- 标签lable里的Def: positive def matrix是作者自己加上去的,为了方便交叉引用,加的标签最好有一定的意义,容易记忆
  $n$阶实对称矩阵$A$为正定的,如果它所对应的二次型$X^T A X$是正定的,即对任意非零的$n$维列向量$X$, 有$X^T A X>0$.
\end{definition}

根据定义\ref{Def: positive def matrix} (注意这里的交叉引用方法), 我们有......

下面是性质,还包含一个列表的使用例子,注意列表编号的格式。
\begin{property}\label{Property: positive matrix}
  如果$A$和$B$都是正定矩阵, 则有:
  \begin{itemize}
    \item[(1)]$A+B$是正定矩阵;
    \item[(ii)]$kA$$(k>0)$是正定矩阵;
    \item[(bla)]blablabla;
    \item[1.]
    \item[i.]
    \item[A.]
  \end{itemize}
\end{property}

以下是一个引理。
\begin{lemma}
  设$E:\Bbb{R}^+ \to \Bbb{R}^+$ ($\Bbb{R}^+=[0,+\infty)$)是一个单调递减的函数且存在常数$T>0$,使得
  $$\int_t^{\infty} E(s) \d s \leq TE(t),\quad \forall t\in \Bbb{R}^+,$$
  则
  $$E(t)\leq E(0) e^{1- \frac{t}{T}},\quad \forall t\geq T. $$
\end{lemma}

下面是一个定理及证明, 注意不等式(\ref{ineq-1})的交叉引用方法.
\begin{theorem}
  设$E$是定义在$[0,\infty)$上的非负递减函数. 如果
  \begin{equation}\label{ineq-1}  %----- 标签lable里的ineq-1是作者自己加上去的,为了方便交叉引用,加的标签最好有一定的意义,容易记忆
    \int_S^{\infty} E(t) \d t \leq CE(S),\quad \forall S\geq S_0,
  \end{equation}
  其中$S_0$, $C$为固定常数, 则
  $$E(t)\leq E(0)e^{1-\frac{t}{S_0+C}},\quad \forall t\geq 0.$$
\end{theorem}

\begin{proof}
  若$0\leq S\leq S_0$, 由(\ref{ineq-1})式可知
  \begin{eqnarray*}
    \int_S^{\infty} E(t)\d t &=& \int_S^{S_0} E(t)\d t +\int_{S_0}^{\infty} E(t)\d t\\
    &\leq &(S_0-S) E(S)+CE(S_0)\\
    &= &S_0 E(S) +CE(S)
  \end{eqnarray*}
  因此, 对$\forall S\geq 0$, 有
  $$\int_S^{\infty} E(t) \d t \leq  (S_0+C)E(S).$$
  由引理3得
  $$E(t)\leq E(0)e^{1-\frac{t}{S_0+C}},\quad \forall t\geq 0.$$
\end{proof}
\begin{remark}
  这里是一个注。
\end{remark}

\begin{theorem}[局部存在性与唯一性, \cite{Liang-Xu-Auto18}]  \label{Thm: local existence}
  假设\textbf{条件} 成立, 则存在依赖于初始二次能量~$\mathscr{E}(0)$ 的~$T>0$ 使得问题在时间区间~$(-\infty, T]$ 上有弱解. 另外, 我们有下面的能量恒等式成立:
  \begin{eqnarray}
    &&\mathscr{E}+\int_0^t\int_\Omega |u_t|^{m+1} \d x\d \tau-\frac12 \int_0^t\int_0^{-\infty}|\nabla w(\tau,s)|_2^2 \mu'(s)\d s\d \tau\nonumber\\
    &&=\mathscr{E}(0)+\int_0^t\int_\Omega |u|^{p-1}uu_t\d x\d\tau,\label{4 quadratic energy identity}
  \end{eqnarray}
\end{theorem}


下面是一个例.
\begin{example}
  这是一个例子.
\end{example}


\subsection{数学公式、符号的例子}

行列式的例子
\begin{equation*}
  |\lambda E- A|=
  \begin{vmatrix}
    \lambda-a_{11} & -a_{12}        & -a_{13} & \cdots & -a_{1n}         \\
    -a_{21}        & \lambda-a_{22} & -a_{23} & \cdots & -a_{2n}         \\
    \vdots         & \vdots         & \vdots  & \ddots & \vdots          \\
    -a_{n1}        & -a_{n2}        & -a_{n3} & \cdots & \lambda -a_{nn}
  \end{vmatrix}.
\end{equation*}

矩阵的例子
\begin{equation*}
  A=(a_{ij})_{n\times n} =
  \begin{pmatrix}
    a_{11} & a_{12} & a_{13} & \cdots & a_{1n} \\
    a_{21} & a_{22} & a_{23} & \cdots & a_{2n} \\
    \vdots & \vdots & \vdots & \ddots & \vdots \\
    a_{n1} & a_{n2} & a_{n3} & \cdots & a_{nn}
  \end{pmatrix}.
\end{equation*}

方程组的例子
\begin{equation*}
  \left\{   %------ 定界符是大括号{。 \left\{   与后面的\right. 对应
  \begin{aligned}
     & u_{tt}-\Delta u+|u_t|^{m-1}u_t=|u|^{p-1}u,\quad (x,t)\in \Bbb{R}^n\times (0,\infty), \\
     & u(0,x)=u_0(x),\quad u_t(x,0)=u_1(x),
  \end{aligned}
  \right.
\end{equation*}

\begin{equation*}
  \left\{
  \begin{array}{rl} %---- 分为两列,第一列右对齐(r=right),第二列左对齐(l=left)
    -x & \text{if } x < 0, \\
    0  & \text{if } x = 0, \\
    x  & \text{if } x > 0.
  \end{array}
  \right.
\end{equation*}


长公式
\begin{eqnarray*}
  J(\psi_t(v);t)&=&\frac{p-2}{2p}(|\nabla \psi_t(v)|_2^2+b|\psi_t(v)|_2^2)+\frac1p I(\psi_t(v);t) \\
  &=&\frac{p-2}{2p}s^2(v;t)\|v\|^2 \\
  &=&\frac{p-2}{2p}(k(t))^{-\frac{2}{p-2}}\|v\|^{\frac{2p}{p-2}}.
\end{eqnarray*}
\begin{eqnarray*}
  &       &\frac{\gamma_a^p\left(2\rho(0)\right)^{1-\frac{p}{2}}}{\left(p-2\right)k(T_3)}\leq T^* \\
  &\leq & T_3:= \frac{8(p-1)(a\l_1+1)\rho(0)}{(p-2)^2[(p-2)(b+\l_1)\rho(0)-p(a\l_1+1)J(u_0;0)]};
\end{eqnarray*}


一个具有斜线表头的表格
\begin{center}
  \begin{tabular}{|c|c|c|} %------ 三列的列表,每列都居中对齐
    \hline
    % after \\: \hline or \cline{col1-col2} \cline{col3-col4} ...
    \diagbox{$X$}{$Y$} & $a$ & $b$ \\
    \hline
    $c$                & 1   & 0   \\
    \hline
    $d$                & 0   & 1   \\
    \hline
  \end{tabular}
\end{center}


三线表
\begin{table}[htbp]
  \centering
  \caption{理论计算得到的PZT运动一个周期内干涉条纹数}
  \label{tab:theory}
  \begin{tabular}{ccc}
    \toprule
    图2.12、 2.13 & 振动幅度$(p-p)$ $(\mu m)$ & 理论计算的条纹数 \\
    \midrule
    (a)           & 0.7$\mu m$                & 2.2              \\
    (b)           & 1.3$\mu m$                & 4.1              \\
    (c)           & 2$\mu m$                  & 6.3              \\
    \bottomrule
  \end{tabular}
\end{table}

